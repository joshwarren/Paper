%\documentclass[a4paper,11pt]{article}
\documentclass[fleqn,usenatbib,useAMS]{mnras}
\usepackage{amssymb}
\usepackage{natbib} % references in text to look like MNRAS/ApJ/A&A
% \usepackage{hyperref} % references become hyperlinks to bibliography
% \hypersetup{
% 	colorlinks,
% 	citecolor=black,
% 	filecolor=black,
% 	linkcolor=black,
% 	urlcolor=black
% }
\usepackage{amsmath} % align enviroment 
% \usepackage{gensymb} % degree symbol in math enviroments
\usepackage{mhchem} % typeset chemistry symbols
% \usepackage{ccaption} % for contcaption to continue figure onto multiple pages
\usepackage[flushleft, referable]{threeparttablex} % Adding notes under tables
\usepackage{multirow} % for multirow command
\usepackage{url}
\urlstyle{same}


\usepackage{array} % For wrapping text in cell in tables
% \newcolumntype{C}[1]{>{\centering\arraybackslash}p{#1}} % centered wrapped text in tables
\newcolumntype{L}[1]{>{\raggedright\let\newline\\\arraybackslash\hspace{0pt}}m{#1}} % 
\newcommand{\bracket}[1]{[#1]} % to be able to use [] in arguments

\defcitealias{Thomas2010}{TMJ} % TMJ alias




\title{An IFS study of Low-powered Radio Galaxies}
\author[J. Warren et al.]{
Joshua Warren,$^{1}$\thanks{Contact e-mail: \href{mailto:joshua.warren@physics.ox.ac.uk}{joshua.warren@physics.ox.ac.uk}}
Martin Bureau,$^{1}$
Bernd Hasemann,$^{2}$
Isabella Prandoni,$^{3}$ \newauthor
Francesco Santoro,$^{3}$
Robert Laing,$^{3}$
Paola Parma,$^{3}$
Hans de Ruiter$^{3}$
and Arturo Mignano$^{3}$
\\
$^{1}$Sub-department of Astrophysics, Department of Physics, University of Oxford, Denys Wilkinson Building, Keble Road, Oxford OX1 3RH, UK\\
$^{2}$European Southern Observatory, Karl-Schwarzschild-Str. 2, 85748 Garching b. München, Germany\\
$^{3}$INAF - Istituto di Radioastronomia, Via P. Gobetti 101, 40129 Bologna, Italy}

\begin{document}
\maketitle

%\date{Oxford, \today} % date is given automatically at top of page in mnras formatting

\begin{abstract}

\end{abstract}



\section{Introduction}
	\label{sec:intro}
	% There are known to be strong scaling relations between the mass of the central supermassive black-hole (SMBH) and the global properties in a given galaxy, such as velocity dispersion, the $M-\sigma$ relation \citep{Ferrarese2000, Gebhardt2000, Graham2011}; Sersic index \citep{Graham2007, Savorgnan2013}; luminosity \citep{Laor2001, McLure2001, Lauer2007, Graham2012}; etc. AGN feedback is understood to be the mechanism by which the SMBH can expand its sphere of influence to galactic scales to affect these relationship. 

	% This is generally understood to be a feedback process, where large amounts of gas within the galaxy cools, falls on the SMBH. Here the black-hole imparts energy to the gas, causing outflows (or jets) and/or thermal emission. We observe this as an Active Galactic Nuclei (AGN) which can have distinctive emission across the electromagnetic spectrum. The exact mechanisms of imparting energy still remains an open question, with much theoretical and observational work ongoing. The outflows can then effect the outer parts of the galaxy, by methods such as fountain effects \citep{}, thermal heating \citep{DeYoung2010} and shocks created by the outflows \citep{}.

	% One identifier of AGNs is a radio signature consistent with Synchrotron emission from the outflow i.e a radio galaxy (RG). These are typically categorized into one of two classes: Fanaroff-Riley (FR) I or FR II \citep{Fanaroff1974}. Of these we focus on the former since these are by far the most common mode in the local universe (possibly as far as $z \sim 1$ \citep{Rigby2008}), and so to discuss radio AGN feedback is to consider feedback from FR I sources \citep{DeYoung2010}. Indeed, while there is broad consensus \citep{Heckman1986, Baum1992} that the brightest RGs (mostly FR IIs) are caused when a massive galaxy merges with a gas rich galaxy (a wet merger) giving a plentiful fuel reservoir to the AGN \citep{Baum1992}, there is much more debate when it comes to low-powered ($P_\mathrm{1.4 GHz} \lesssim 10^{24.5} \, \mathrm{W Hz^{-1}}$) RGs (mostly FR I). Suggestions fall into two categories: an extrapolation of radio-loud galaxies (i.e. a merger with either small gas reservoirs or low efficiency accretion onto the SMBH) or a highly efficient accretion of gas from secular origins (such as Bondi accretion of the hot X-ray component of the Inter-Stellar Medium (ISM) \citep{Allen2006} or from existing cold gas reservoirs \citep{Prandoni2010}). This project aims to add to this discussion using Integral Field Unit (IFU) observations in the visible band of nearby radio galaxies and comparing this with Atlas-3D sample \citep{Cappellari2011} as a control sample.

% More to add to intro - e.g. summery of paper 

\section{Sample Description}
	\label{sec:samp}
	The Parkes 2.7 GHz survey \citep{Ekers1989} was used as a parent sample, as it contains radio sources (down to a flux density limit of 250\,mJy at 2.7\,GHz) with an optical counterpart (with an $V$-band apparent magnitude of $m_V \le 17.0$). Host galaxies with a redshift greater than 0.03 were discarded. It is worth noting that both the radio and optical limits are apparent measurements, and as such this is not a volume-limited sample.

	\begin{table*}
		\centering
	\begin{threeparttable}
		\caption{Key sample characteristics of the Southern Sample galaxies.}
		\label{tab:sample}
		\begin{tabular}{l l r r r l}
			\hline
			\hline
			Host galaxy	& Radio source 	& Redshift	& $S_\text{1.4\,GHz}$	& $m_K$ & Dust morphology\\
						& (PKS) 		& 			& (mJy) 			& (mag)	&\\
			\hline 
			ESO 443-G024 & 1258-321 	& 0.01703	& $1387 \pm 45$		& 8.51 & no dust\tnote{a}	\\ 
			IC 1459 	& 2254-367 		& 0.00565 	& $1280 \pm 45$		& 6.81 & dust lane\tnote{b}	\\
			IC 1531 	& 0007-325 		& 0.02553 	& $388 \pm 12$		& 9.55 & --					\\
			IC 4296		& 1333-\leavevmode\phantom{0}33 		& 0.01248 	& $1342 \pm 41$		& 7.50 & edge-on disc\tnote{b} \\
			NGC 612 	& 0131-\leavevmode\phantom{0}36 		& 0.02954 	& $585 \pm 18$		& 9.58 & dust lane\tnote{c}	\\
			NGC 1316 	& 0320-\leavevmode\phantom{0}37 & 0.00591 	& $255 \pm 10$		& 5.59 & dust patches\tnote{b} \\
			NGC 1399 	& 0336-\leavevmode\phantom{0}35 & 0.00472 	& $209 \pm \leavevmode\phantom{0}7$	& 6.31 & no dust\tnote{b}	\\
			NGC 3100 	& 0958-314 		& 0.00879 	& $530 \pm 16$		& 8.08 & dust lane\tnote{d}	\\
			NGC 3557 	& 1107-372 		& 0.01016 	& $484 \pm 16$		& 7.203 & face-on disc\tnote{b}\\
			NGC 7075 	& 2128-388 		& 0.01819 	& $837 \pm 28$		& 9.56 & --					\\
			--			& 0718-\leavevmode\phantom{0}34 		& 0.02897 	& $1119 \pm 41$		& 9.97 & dust patches\tnote{e} \\
			\hline
			\hline
		\end{tabular}
		\begin{tablenotes}
		\footnotesize
		\note Col.\,1: host galaxy name. Col.\,2: radio source name from Parkes Southern Radio Source Catalogue. Col.\,3: redshift. Col.\,4: National Radio Astronomy Observatory (NRAO) Very Large Array (VLA) Sky Survey (NVSS) 1.4 GHz flux density \citep{Condon1998}. Col.\,5: Two Micron All Sky Survey (2MASS) $K$-band apparent magnitude (with errors of $\pm 0.025$ mag; \citealt{Skrutskie2006}). Col.\,6: dust morphologies from (a) \citet{Govoni2000}, (b) \citet{Lauer2005}, (c) \citet{Bettoni2001}, (d) \citet{Sandage1979}, (e) \citet{Colbert2001}. 
		\item We hereafter refer to the last galaxy by its PKS name only.
		\end{tablenotes}
	\end{threeparttable}
	\end{table*}

	These criteria lead to a sample of 11 galaxies hereafter referred to as the Southern Sample. The key observed characteristics are summarised in Table \ref{tab:sample}.
	%, while the derived characteristics are listed in Table \ref{tab:sampleDerived}. 
	All galaxies have a FR I radio morphology according to the \citet{Fanaroff1974} scheme; the bulk of the radio emission is centrally-concentrated, as opposed to the FR II morphology where the brightest points are at the ends of the radio lobes. 
	
	The sample was initially observed with the Atacama Pathfinder Experiment \citep[APEX; ][]{Gusten2006}, with detections claimed for the \ce{^{12}CO(2-1)} transition for all galaxies. Some galaxies had extremely large line widths (up to a full-width half-maximum of $904 \, \mathrm{km \, s^{-1}}$ for IC 4296), but most showed a flat or double-horn spectrum indicative of ordered rotation \citep{Prandoni2012}.


\section{Observing strategy and Data Reduction}
	\label{sec:obs}
	The sample was observed with the Visible Multi-Object Spectrograph (VIMOS), mounted on UT3 on the VLT in Paranal \citep{LeFevre2003}. All observations were taken with a spatial sampling of $0\farcs67$, using the high-resolution blue grism yielding a wavelength range of 3700-5520\,\AA\ with a spectral resolution of 1440 and sampling of 0.71\,\AA\,$\mathrm{pix^{-1}}$. Each object was observed with a total integration time of $\approx 100$\,min on-source, equally spread over three observing blocks (OBs). Each OB also contained all the necessary observation for standard IFS data reduction. 

	VIMOS has several well-known though poorly-understood technical issues. These include several low transmission (bad) fibres, strong flexure, contamination from adjacent spectra on the CCD (cross-talk) and large differences in sensitivity across its 4 quadrants. To address these issues, we adopted a data-reduction pipeline produced using \textsc{Py3D}, a suite of programs based on the \textsc{python} versions of those developed for the Calar Alto Legacy Integral Field spectroscopy Area survey \citep[CALIFA;][]{Sanchez2012, Husemann2013} but later updated for VIMOS by \citet{Husemann2014}. \textsc{Py3D} applies all standard IFS data-reductions steps: bias subtraction, spectrum identification, flatfielding and wavelength calibration. In addition, the data was flux calibrated using publicly available observations of the spectro-photometric standard star Feige 110 provided by ESO. All steps include methods to account for the bad fibres, flexure and cross-talk within VIMOS. 

	Following this, we noted that the cubes where still not properly calibrated. Three main issues remain: the quadrants have different intensities, clearly showing uncorrected throughput differences; diagonal intensity stripes are present in the reconstructed images at all wavelengths; and spectral features are observed \citep{Jullo2008} that are visually similar to fringe patterns caused by interference within the CCD between incident light and light reflected from the interfaces between layers of the CCD materials. \citet{Lagerholm2012} present ad-hoc methods to correct for these issues, which we here adopt. These additional steps mean that the datacubes are not perfectly flux-calibrated, but all the corrections are multiplicative and thus will not effect equivalent width or line ratio measurements. From comparisons to the MUSE data (see Section \ref{sec:MUSE}), we then estimate the flux calibration of the resulting datacubes. 

	The variance spectra are propagated throughout the data-reduction pipeline (including these ad-hoc corrections) and are square-rooted at this point, to be used as noise inputs in the following analyses (see Section \ref{sec:analysis}).


	\subsection{MUSE Archival Data}
		\label{subsec:MUSE}
		Four of the sample galaxies are in the Multi-unit Spectroscopic Explorer (MUSE) archive: IC 1459, IC 4296, NGC 1316 and NGC 1399. We include these in our analysis as they have better (spatial and spectral) resolution and sampling, a larger field-of-view and larger wavelength range than the VIMOS data. 

		We used the pre-reduced (Phase 3) datacubes from ESO. These were generally of sufficient quality for our purpose, except in both IC 1459 and IC 4296 the sky appears to have been over-subtracted. To remove this over-subtraction, we developed our own pseudo-sky subtraction routine. A median spectrum was taken from four $20 \times 20$\,spaxel regions, one in from each spatial corner of the ESO reduced cube. After checking that no stellar continuum could be fit to the medium sky spectrum (i.e.\ that very little light from the galaxy is contaminating the pseudo-sky regions), this median sky spectrum was subtracted from each spaxel in the cube. 

		Finally, we trimmed all the MUSE cubes to the central $30\arcsec \times 30\arcsec$ ($150 \times 150$\,spaxels) only, to (a) avoid the regions used for the pseudo-sky spectrum and (b) reduce the computing resources required for spatial binning (see Section \ref{sec:analysis}). In any case, the bins would be so large in the outer parts as to be effectively useless (see Fig\,\ref{fig:egSNR}).

\section{Data Analysis}
	\label{sec:analysis}
	From this point on, the VIMOS and MUSE datasets are treated almost identically, the only difference (other than the different spectral ranges and resolutions) being that the MUSE datacubes are spatially binned to a higher signal-to-noise ratio (S/N) using Voronoi binning\footnote{\url{http://www-astro.physics.ox.ac.uk/~mxc/software/}} \citep{Cappellari2003}. For our purposes, we define the `signal' and `noise' of each spaxel as the median value of its spectrum and noise spectrum, respectively. We require a S/N of 30 for all VIMOS datacubes and 50 for the IC 1459 and IC 4296 MUSE datacubes. The NGC 1316 and NGC 1399 MUSE datacubes were binned to a S/N of 50 for the analysis of the stellar kinematics and 100 for the analysis of the emission line kinematics and stellar populations.

	Next we implement a routine to find optimal initial guesses of the redshift and velocity dispersion for each galaxy, to be used when fitting individual bins, by means of iterative fits to the spatially collapsed (i.e. summed across both spatial dimensions of the datacube) spectrum. We use the penalized-fitting (\textsc{pPXF}) routine\footnote{\url{http://www-astro.physics.ox.ac.uk/~mxc/software/}} of \citet{Cappellari2004} and \citet{Cappellari2016a} with the Medium-resolution Issac Newton Telescope (INT) Library of Empirical Spectra (MILES) library \citep{Sanchez-Blazquez2006, Falcon-Barroso2011a} to find the best-fitting line-of-sight velocity distribution (LOSVD; as parametrised by the Gaussian parameters $v$, the mean velocity and $\sigma$, the velocity dispersion). The final mean velocity of a given galaxy, is used as its precise spectroscopic redshift and is given in Table \ref{tab:sample}. Beyond this point, each bin is analysed independently.

	\subsection{Stellar Kinematics}
		\label{subsec:starKin}
		Each bin is fit independently again using \textsc{pPXF} with the MILES library. For the fit regions around potential emission lines are masked, and a fourth order polynomial additive continuum correction is used. A Monte--Carlo (MC) estimate of the uncertainties is used. 

	\subsection{Ionized Gas Distribution and Kinematics}
		\label{subsec:EmissionLines}
		We fit the ionized gas fluxes and kinematics from the emission lines listed in Table \ref{tab:EmissionLine}, using Gaussian templates are used to independently fit emission lines from the interstellar medium (ISM), with a single LOSVDs, independently of the stellar kinematics. However, to minimize template mismatch, whereby emission lines are erroneously fitted to the edges of badly modeled and thus subtracted absorption features, we follow the method set out in \citet{Sarzi2005}, whereby the stellar component is first fitted as above with masked potential emission line regions, followed by the unmasking and fitting of the [\ion{O}{iii}] doublet which sets the kinematics for all the emission lines and finally the amplitude of the remaining emission lines. Some of the forbidden emission lines have the ratio of their amplitudes fixed (see Table \ref{tab:EmissionLine}, Col.\,4).

		\begin{table}
	 		\centering
	 	\begin{threeparttable}
	 		\caption{Emission lines considered in the \textsc{pPXF} fits.}
	 		\label{tab:EmissionLine}
	 		\begin{tabular}{l c c c}
	 		\hline
	 		\hline
	 		Emission & Rest-frame & Doublet rest-frame & Amplitude  \\
	 		Line & wavelength & wavelength & ratio \\
	 		 & (\AA) & (\AA) \\
	 		\hline
	 		\bracket{\ion{O}{ii}} 	& 3726.03 & 3728.82 & Free \\
	 		H\,$\delta$ 	& 4101.76 & -- & -- \\
	 		H\,$\gamma$ 	& 4340.47 & -- & -- \\
	 		H\,$\beta$ 		& 4861.33 & -- & -- \\
	 		\bracket{\ion{O}{iii}}	& 4958.92 & 5006.84 & 0.35 \\
	 		\bracket{\ion{N}{i}} 	& 5199.36 & 5201.86 & 0.65 \\
	 		\bracket{\ion{O}{i}} 	& 6300.30 & 6363.67 & 0.33 \\
	 		\bracket{\ion{N}{ii}} 	& 6548.03 & 6583.41 & 0.34 \\
	 		H\,$\alpha$ 	& 6562.30 & -- & -- \\
	 		\bracket{\ion{S}{ii}} 	& 6716.47 & 6730.85 & Free \\
	 		\hline
	 		\hline
	 		\end{tabular}
	 		\begin{tablenotes}
	 		\note Col\,1: Emission line name. Col\,2: Emission line rest-frame wavelength. Col\,3: Doublet rest-frame wavelength for forbidden lines. Col\,4: The fixed ratio of the amplitudes of the lines within a doublet. `Free' indicates the amplitudes of both doublet constituents are fit independently of each other. 
	 		\end{tablenotes}
	 	\end{threeparttable}
	 	\end{table}


		

	\subsection{Absorption Line Strengths and Stellar Populations}
		\label{subsec:absorption}
		Due to the lack of calibration observations of Lick/Cassegrain Image Dissector Scanner spectrograph system (hereafter Lick/IDS system; \citealt{Faber1985, Worthey1994}) we use the line index system (LIS) of \citet{Vazdekis2010} instead. This relies on flux calibrations instead of correcting to the continuum shape and wavelength dependent resolution of the Lick/IDS system. We use the index definitions of \citet{Trager1998}, measuring the following indices: G4300, Fe4383, Ca4455, Fe4531, H\,$\beta$, Fe5015, Mg\,b, Fe5270, Fe5335, Fe5406, Fe5709, Fe5782, NaD, TiO1 and TiO2. TiO1 and TiO2 are considered molecular absorption lines and are measured in magnitude units, while all others are atomic and measured in angstroms. 

		Quoted line strengths are corrected for both emission line and the effect of the velocity dispersion of the stars. These corrections are detailed below.

		\subsubsection{Removing Emission Lines}
			Emission lines are removed from the spectra by subtracting the best-fitting emission lines from Section \ref{subsec:EmissionLines} from the data and using only the stellar templates (and continuum correction) for reconstructing the best-fit for use in the velocity dispersion correction (see below).

		\subsubection{Correcting for Stellar Velocity Dispersion}
			We correct for the effect of the different velocity dispersions of the stars which varyingly spread absorption features such that differing fractions of the absorption is outside of the central bandpasses. To do this, we create a best-fitting spectrum (as in the previous step) which is \textit{not} convolved with the best-fitting LOSVD. A given index, $I$, is then measured for both this `unconvolved' spectrum ($I^\text{unc}$) and the convolved best-fitting spectrum ($I^\text{conv}$), and the ratio of these two indices is used as a multiplicative correction factor for the index measured ($I^\text{obs}$), such that the corrected index ($I^\text{corr}$) is given by
			\begin{equation}
				I^\text{corr} \equiv \frac{I^\text{unc}}{I^\text{conv}} I^\text{obs} \, .
			\end{equation}
			These corrected indices are those reported and discussed in the sections below.

		\subsubsection{Best-fitting Stellar Populations}
			\label{subsubsec:stellarPop}
			In each bin, using the measured absorption line strengths, we identify the best-fitting single stellar population (SSP) model from the models of \citeauthor{Thomas2010} (\citeyear{Thomas2010}; hereafter \citetalias{Thomas2010}) using \textsc{emcee}, a \textsc{python}, affine-invariant Markov chain Monte--Carlo (MCMC) fitting package \citep{Foreman-Mackey2013}. The \citetalias{Thomas2010} models are chosen because they are based on high-resolution, flux-calibrated spectra and therefore not requiring absorption line strength measurement to be in the Lick/IDS system. 

\section{Kinematics}
	\label{sec:kine}
	% The kinematics of the sample are classified according to the Regular-Rotator/Non Regular-Rotator (RR/NRR) regime given in \citet{Krajnovic2011}, Fast/Slow Rotator (FR/SR) regime given in \citet{Cappellari2016} (originally defined by \citet{Emsellem2011}, but later refined by \citet{Cappellari2016}). Beyond this attempts have been made to use the kinematic features as defined in \citet{Krajnovic2011}, however the quality of the data has meant that many have had to be classified by eye as the artifacts from the VIMOS quadrants confuse any ellipse fitting methods. 

	% The following briefly describe the observations and results of each of the sample. %It is also worth noting that many maps have been clipped in the color axis, to allow the more detailed structures to not be overwhelmed by the extremes. 

	% \textbf{IC 1459} is known to contain a KDC. This is clearly seen in both the VIMOS and MUSE velocity maps (Fig. \ref{fig:stellar_vel}, \ref{fig:MUSEstellar_vel}). It is also known to have ionized gas counter rotating to the decoupled core. This is again seen by comparing Fig. \ref{fig:stellar_vel} with \ref{fig:gas_vel} and \ref{fig:MUSEstellar_vel} with \ref{fig:MUSEgas_vel}. From the MUSE velocity maps it can be seen that the gas is not coupled to the outer parts of the galaxy either. The KDC appears to be embedded in a slow rotator, though the KDC contaminates the $\lambda_{Re}$ measurement such that it is classified as a Fast Rotator. This is consistent with Section \ref{subsec:KDCsize}.

	% \textbf{IC 1531} seems to contain a KT. This galaxy has a very limited detection of ionized gas concentrated in the center (with the exception of [NI] (Fig. \ref{fig:NI_eqW}), which is more dispersed).

	% \textbf{IC 4296} appears to have KT, though this may be a quadrant feature. There is potentially 2 peaks in all of the gas intensity maps, one at the center of the galaxy and one to the south-east which is not seen in the image (total flux/collapsed cube).

	% \textbf{NGC 0612} has a large dust lane to the east of the apparent center of the galaxy perpendicular to the axis of rotation. The dust lane is also seen as a lower velocity dispersion (Fig \ref{fig:stellar_sigma}). Dust lanes generally imply a disky galaxy: indeed this seems to be the case here as the dust lane is aligned with the plane of the disk. The kinematic maps show NF. The dust lane also contains large amounts of gas. 

	% \textbf{NGC1316} (Fornax A) was not observed with VIMOS, however the MUSE maps show it to have a clear rotation signature (Fig. \ref{fig:MUSEstellar_vel}), though disturbed kinematics (e.g. Fig. \ref{fig:MUSEstellar_sigma}). The ionized gas is very clumpy and has no rotation (Fig. \ref{fig:MUSEgas_vel}).

	% \textbf{NGC 1399}, the central galaxy of the Fornax Cluster \citep{Jordan2007}, is known to have kinematic twist (see MUSE map in \citet{Zieleniewski2017}) on the scale of our MUSE field of view (reduced to 30"). Very little ionized gas was detected. 

	% \textbf{NGC 3100} is has NF in the stellar kinematics, however there is significant amount of ionized gas, which seems to be split into two clouds. This is most obviously seen in Fig. \ref{fig:Hbeta_eqW}. The gas also seems to have a non-standard rotation, possibly linked to its spacial distribution. 

	% \textbf{NGC 3557} is known to be FR with very high velocities, especially considering it's size, with NF. In our maps, there some significant quadrant effects. NGC 3557 also has a very dispersed, non-centrally concentrated H$_\mathrm{\beta}$ distribution. 

	% \textbf{NGC 7075} appears to have NF, though with quite slow velocities. There is some H$_\mathrm{\beta}$ detected at the very center of the galaxy. 

	% \textbf{PKS 0718-34} is a KDC, though S/N issues mean that as in IC 1459, the galaxy cannot be seen beyond the core. It has very little gas detected, though the often faint $H_\mathrm{\delta}$ line is detected. [NI] is redshifted out of the VIMOS spectral range.

	% \textbf{ESO 443-G024} is consistent with a NR. It has a very dispersed H$_\mathrm{\beta}$ (similar to NGC 3557 (Fig. \ref{fig:Hbeta_eqW})).

	% \subsection{In/Out-flows}
	% 	\label{subsec:inflows}


\section{Absorption Line Strengths and Stellar Populations}
	\label{sec:stellarPop}
	% Absorption line strengths were measured using \textsc{python} code developed by \_\_\_\_\_\_. We use the bandpasses defined in \_\_\_\_\_\_. 

	% We measure the line strengths in several steps:
	% \begin{enumerate}
	% 	\item Firstly we subtracted any fitted emission lines from the spectra and de-redshift to the rest frame. This included removing emission lines that we did not consider a detection. The spectrum is then convolved to a FWHM of 8.4\AA in order to match the LIS system \citep{Vazdekis2010}. The absorption line strengths of this stellar-only spectra was measured. This is the principle observation, $EW_{i,obs}$.
	% 	\item From the reported weightings of the stellar templates (and multiplicative polynomials) we built a spectra that is unconvolved with the LOSVD, which we measure the line strength for. This is the unconvolved bestfit and the measured line strengths are $EW_{i,unc}$.
	% 	\item We measure the line strengths for the best-fitting spectrum (with emission lines subtracted and de-redshifted). This is the convolved bestfit with line strengths being $EW_{i,con}$.
	% 	\item Finally, we use the ratio of the line strengths of the unconvolved and convolved bestfits as a correction factor for velocity dispersion. 
	% \end{enumerate}
	% The final value is therefore:
	% \begin{equation}
	% EW_i = \frac{EW_{i, unc}}{EW_{i, con}} EW_{i, obs}
	% \end{equation}

	% To test the routine we measured the absorption line strength for the SAURON project \citep{Kuntschner2006} galaxies within a radius of $R_e/8$ and compare to the measurements found by \citet{Vazdekis2010}. We found a mean difference of 0.29 \AA and a spread of 0.61 \AA between our measurements and \citet{Vazdekis2010}.

	% \begin{table}
	% 	\caption{Comparisons to the literature}
	% 	\label{tab:litAbsorption}
	% 	\begin{tabular}{c c c c}
	% 		\hline
	% 		\hline
	% 		Index 		& N$_{gals}	& Offset 	& Dispersion \\
	% 					& 			& $\AA$		& $\AA$ \\
	% 		\hline
	% 		\multicolumn{4}{c}{\citet{Vazdekis2010}} \\
	% 		\hline
	% 		H$_\beta$ 	& 46		& -0.02		& 0.25	\\
	% 		Fe5015		& 46		& 0.66		& 0.34	\\
	% 		Mg$_b$ 		& 46		& 0.06		& 0.33	\\
	% 		\hline
	% 		\multicolumn{4}{c}{\citet{Rampazzo2005} (VIMOS)}
	% 		\hline
	% 		G4300 		& 3 		& 			& \\
	% 		Fe4383 		& 3 		& 			& \\
	% 		Ca4455 		& 3 		& 			& \\
	% 		Fe4531 		& 3 		& 			& \\
	% 		H$_\beta$ 	& 3 		& 			& \\
	% 		Fe5015 		& 3 		& 			& \\
	% 		Mg$_b$ 		& 3 		& 			& \\
	% 		\hline
	% 		\multicolumn{4}{c}{\citet{Rampazzo2005} (MUSE)}
	% 		\hline

	% 		\hline
	% 	\end{tabular}
	% \end{table}



% \section{Discussion}
% 	\label{sec:discuss}
% 	\subsection{Selection Bias}
% 	\label{subsec:bias}
% 	It was originally intended that this sample should be volume limited, however the selection criteria for the parent sample \citep{Ekers1989} are based on apparent qualities rather than absolute (radio flux density and apparent V-band apparent magnitude). The effect is shown as redshift bias in figure \ref{fig:redshift_bias}. This shows a clear redshift dependence in $P_\mathrm{2.7 GHz}$ as well as (with the exception of NGC 612, a known radio galaxy anomaly) a bias towards low redshift for higher $\lambda_{Re}$. Interestingly, though $M_k$ is well spread in redshift. 

% 	\subsection{BPT}
% 	\label{subsec:BPT}
% 	The BPT diagrams can be useful diagnostics for the dominant source of ionization. In the age of the IFU, the diagrams can be used in a spatially resolved manner to produce maps of ionizing source. 

% 	It is worth taking a moment to 

% 	Much of the current literature use the BPT diagram as an absolute diagnostic tool, were everything below the \_\_\_\_\_\_\_\_\_\_ line is assumed to be entirely due to star formation, while everything above both the \_\_\_\_\_\_ and the \_\_\_\_\_ lines is due to a Seyfert 2 type AGN, and the remaining region is due to a LINER type AGN. This is not how it was originally intended to be used. In fact the \_\_\_\_\_\_\_\_ line marks the upper limit of where star formation \textit{could} be the sole source of ionization. Above the \_\_\_\_\_ line is where an AGN \textit{could} be the sole source of ionization. The region between is where is were either \textit{could} be the sole source of ionization. 







% 	\subsection{KDC stellar populations}
% 	\label{subsec:KDCsize}
% 	The relationship between the size of a KDC, it's age and the properties of the host galaxy was first set out in \citet{McDermid2006}. Here is was found that two populations of KDCs existed: the first was very small, but could be any age, and was typically embedded in a fast rotator host. The second was any size, but old, and was embedded in a slow rotator.  


% 	Figure \ref{fig:KDCsize} shows all KDCs in our sample fit the relationship found by the SAURON group: they are all large and old. This adds weight to the suggestion that they are embedded in intrinsically slow rotators. Given the distance of our objects, we lack the resolution to properly resolve the small KDCs with in the fast rotators, despite the fast rotators being biased to be observed at a lower redshift.



% \section{Conclusion}
% 	\label{sec:conc}
% 	We have presented the VIMOS and MUSE observations of our sample of low-powered radio galaxies. We find a diverse range of kinematic features and classifications as well as varying detection rates in both brightness and extent. 



\section*{Acknowledgments}
% \addcontentsline{toc}{section}{Acknowledgments} % add to contents - not sure if MNRAS want this...





%\FloatBarrier
\bibliographystyle{mnras}

\bibliography{refs}{}

\appendix

% \input{appendix.tex}

\end{document}